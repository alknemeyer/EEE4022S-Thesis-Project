\section{Introduction}
\subsection{Background to the study}

While the ability to autonomously track an object has a wide range of applications, it is useful to limit oneself to a particular need in order to produce quantifiable success/failure metrics.

The UCT mechatronics laboratory studies cheetahs with the aim of reproducing their extraordinary manoeuvrability in two and four legged robots. For this, they require footage of cheetahs as they run. Currently, they place multiple static pairs of cameras in order to continuously film an animal as it runs around. Having multiple cameras requires significant amounts of setup time, effort into synchronizing the footage and a total higher price as the cost of multiple cameras is expensive. In addition, to time spent recording the animal, the cameras must be set to wide-angle which introduces distortions.


\subsection{Objectives of the study}
The objective of the study is to design and implement a system which can autonomously identify and track a fast-moving object using modern computer vision methods. A successfull design would be cheaper overall, track the object for longer, and not require wide-angle cameras.


\subsection{Scope and limitations}
Due to the limited time and finances in an undergraduate thesis, the system will not be test on actual cheetahs during the project. Instead, it will be tested on humans and trained dogs with the hope that this sufficiently mimics the scenariou of tracking cheetahs (or any other fast-moving object, for that matter).

In addition, the study will be limited to tracking only one object in the frame at a time.


\subsection{Plan of development}
The development schedule was as follows:

\begin{enumerate}
\item Design the structure of the system as a whole.
\item Set up the raspberry pi, installing relevant software and testing the camera module.
\item Choose a neural network architecture and deploy it onto the accelerator.
\item Design and 3D print a camera gimbal, and procure the necessary controller and actuators.
\item Rewrite the gimbal controller communication standard in python to allow the raspberry pi to send and receive commands.
\item Model the system, and then design and implement an Extended Kalman Filter (EKF).
\item Implement a parallel process-based data pipeline which takes a photo on the picam, preprocessess it, passes it through the neural accelerator, retrieves the results, passes the results into the EKF and then commands the gimbal motors to re-orient the camera appropriately.
\item Test final the system
\end{enumerate}

\subsection{Report outline}


