\documentclass{article}

\usepackage{graphicx}

%-----------------------------------------------------------------------
%-----------------------------------------------------------------------
%-----------------------------------------------------------------------

\begin{document}
\title{Alex Thesis Project}
\author{Alexander Knemeyer}

\maketitle

\pagenumbering{gobble}  % turn off page numbering

\newpage

\pagenumbering{roman}

\begin{abstract}
% [Explain the need/uses]
Many systems require a camera to identify the objects in it's field of view, and then move the camera to keep it aimed at a certain type of object. Examples of this need include filming cheetahs to further research into bio-mimicry, having a quadcopter record and follow a sportsperson as they move around and helping wildlife parks to spot and track poachers before they can cause harm.

% [Paragraph on computer vision]
Luckily, there has been a recent surge in performance for computer vision techniques. Neural networks often achieve higher accuracy scores than humans in tasks such as object detection. The technology is in general very accessible, with a strong culture of open-research and open-source.

% [Paragraph on raspberry pi + NCS]
However, image recognition using neural networks is an extremely compute intensive task. The nature of the computation suits parallelisation, but inferring a result from an underpowered mobile device (such as a raspberry pi, which is a type of single board computer) can take a full second or longer. Again, there exists a solution: neural accelerators, which are essentially a GPU on a USB stick. These enable object recognition in near real-time.

% [Paragraph on EKF]
Despite the speedups offered by neural accelerator sticks, the maximum number of inferences possible per second is still somewhat low. In this respect, the Kalman Filter can be useful. It enables the computer to predict where the tracked object will be in between updates from the neural network.

Thus, the aim of this project is as follows: using a modern neural network architecture, identify and track cheetahs from a mobile platform. The computing power is limited (a Raspberry Pi and Movidius Neural Compute Stick) which results in slow update rates. In addition, only position is measured. Thus, an Extended Kalman Filter is used to help ensure robust and smooth tracking of the target.
\end{abstract}

\newpage

%-----------------------------------------------------------------------
%-----------------------------------------------------------------------
%-----------------------------------------------------------------------

\section*{Terms of Reference}

%-----------------------------------------------------------------------
%-----------------------------------------------------------------------
%-----------------------------------------------------------------------

\section*{Acknowledgement}
thanks Callen and Amir \\
thanks Justin Pead \\
thanks Stacey for laser cutting \\
thanks family and friends

\newpage

%-----------------------------------------------------------------------
%-----------------------------------------------------------------------
%-----------------------------------------------------------------------

\tableofcontents
\newpage

\listoffigures
\listoftables

\newpage

\pagenumbering{arabic}

%-----------------------------------------------------------------------
%-----------------------------------------------------------------------
%-----------------------------------------------------------------------

\chapter{Introduction}
This report details the research, design, implementation and testing of an automatic object detection and tracking system. It was made as a part of a final year undergraduate project.

\section{Background to the study}
For years, some researchers have turned to bio-mimicry for insight and inspiration into the design of robots. One such example is the University of Cape Town (UCT) mechatronics laboratory's Rapid Acceleration and Manoeuvrability group, which studies cheetahs with the aim of reproducing their extraordinary manoeuvrability in two and four legged robots \cite{website:uct_mechatronics}. To understand the dynamics of cheetah movement, they require footage of animals as they start and stop a sprint. A requirement of their research is that the cameras must be placed in pairs to achieve depth perception~\cite{website:depth_perception_from_stereo_camera}.

The researchers currently place multiple static cameras, with each pair covering a different area, in order to continuously film an animal through a full sprint. An illustration of this can be seen in Figure~\ref{fig:multiple_gopro_pairs}. \\

\begin{figure}[h!]
  \centering
  \includegraphics[width=\textwidth]{multiple_gopro_pairs}
  \caption{\label{fig:multiple_gopro_pairs} An illustration of the camera set up in a typical animal movement study. Cheetah clipart from Change.org \cite{website:pic_of_cheetah}.}
\end{figure}

Unfortunately, this setup is not without its disadvantages: having multiple cameras requires extra setup time, effort into synchronizing the footage, money spent on equipment and extra work in keeping all devices charged. To maximise the time spent recording a sprint, the cameras are usually set to wide-angle mode. This introduces distortion into the recordings \cite{website:camera_distorion}.

This resulted in the following question: what if the cameras could instead rotate to track the cheetah as it moves? This way only a single pair of cameras would be required. An illustration of this in two moments of time is shown in Figure~\ref{fig:single_gopro_pair}.

\begin{figure}[h!]
  \centering
  \includegraphics[width=\textwidth]{single_gopro_pair}
  \caption{\label{fig:single_gopro_pair} The proposed solution: a single pair of cameras tracking an animal as it moves.}
\end{figure}

The researchers specified that such a tracking system should,

\begin{itemize}
	\item Not require a beacon, light or any other object to be placed on the object,
	\item Locate the object with high accuracy,
	\item Be able to differentiate between classes of objects, while treating different instances of the same class equally (in other words, locate \emph{any} cheetah regardless of its specific spot pattern),
	\item Have a low enough latency that the object can't escape the field of view of the cameras before a result is reached,
	\item Use a provided Raspberry Pi model 3 B as the main computing platform so as to remain mobile,
	\item Work robustly - this means locating the regardless of its orientation, lighting, specific shape, background, and so on - and finally,
	\item Be able to track any other object with minimal extra work or design required, should the focus of research be changed.
\end{itemize}

After comparing existing object tracking technologies, it was decided that the only way to robustly track the cheetahs in this scenario would be to use a neural network. This decision is elaborated on in section \ref{sec:compare_cv_techniques} of this report. The comparison showed that only machine learning techniques can accurately discriminate between a range of classes of objects using affordable and accessible sensing hardware alone (such as low-cost cameras).


\section{Other object tracking systems}
There are a large number of applications for a system which can track objects in real time using computer vision techniques. Some include:

\begin{itemize}
\item A robot which can sense nearby humans and make sure to not hurt them, or decide on a course of action which depends on the type of object in front of it.
\item A small autonomous airplane which patrols the skies above a national park, and broadcasts the location of any potential poachers that it finds.
\item A camera system which automatically tracks a ball or the centroid of the position all players, helping to record sports games.
\end{itemize}

Object tracking systems have already been incorporated into existing commercial products. One example is the "Mavic" drone from DJI, which can autonomously track humans \cite{website:DJI_mavic}. This allows people to launch the drone and have it autonomously follow them as they move around.

Another example is EyeCloud, a company that has developed a security system which covers a wider area than a single static camera could, and only sends an alert when it detects humans \cite{website:eyecloud}. This prevents false alarms from sounding when safe classes of objects, such as dogs, enter the field of view.



\section{Objectives of the study}
Based on the problem stated by the UCT researchers, the objective of the study is to design, build and test a system which can autonomously identify and track a fast-moving object using modern computer vision methods. A successful design would be able to rotate a pair of cameras to continuously aim at an object, track the object for longer than a stationary camera pair could, and not require the cameras to be set to distorting wide-angle modes.

Note that, while the ability to autonomously track an object has a wide range of applications, it is useful to limit oneself to a particular need in order to produce quantifiable 'success or failure' metrics. Thus, this project and report are mainly about solving a specific issue, with the premise that the project can be used for other applications with minimal modification. Discussions on literature and other applications will treat the system as a general purpose object tracker.



\section{Scope and limitations}
Due to the limited time and finances in an undergraduate thesis, testing the tracking system on actual cheetahs was not part of the agreed scope of this project. Instead, it was tested on moving humans in complex environments with the hope that this sufficiently mimics the scenario of tracking cheetahs (or any other fast-moving object) in the wild.

Additionally, the study was limited to tracking only one object in the frame at a time. This is because it is not immediately obvious what should be done when more than one object is present in the frame.


\section{Plan of development}
A development schedule was devised to decide how much time could be allotted for each component of the final system. It was as follows,

\begin{enumerate}
\item Create a rough design for the system as a whole to understand which skills and components would be necessary.
\item Spend a month learning relevant new theory, mostly in the field of computer vision.
\item Set up the main computing platforms, installing relevant software and testing the camera module, over the course of a week.
\item Optimize a cutting edge neural network for the specific scenario in this project over the course of two weeks. Failing that, choose an existing neural network architecture.
\item Compile the neural network and deploy it onto a neural accelerator stick, writing appropriate software, over the course of a week.
\item Use two weeks to design and 3D print a camera gimbal, and procure the necessary controller and actuators.
\item At the same time, rewrite the gimbal controller communication standard in python to allow the raspberry pi to send and receive commands.
\item Spend a week modelling the expected movement of the tracked object, and designing and implementing an appropriate Kalman Filter. Update this based on preliminary test results.
\item Spend two weeks implementing a parallel process-based data pipeline which takes a photo on the Pi camera, pre-processess it, passes it through the neural accelerator, retrieves the results, passes the results into the Kalman Filter and then commands the gimbal motors to re-orient the camera appropriately. 
\item Test final the system over the course of a week, tweaking the design based on results.
\item Use the last three weeks to write a report which summarizes the entire project, based on small notes taken during the course of the project.
\end{enumerate}



\section{Report outline}

The report begins with a review of the current literature surrounding object tracking. As the system combines a number of technologies, it also presents previous knowledge on all of the components which ultimately formed a part of final system.

Next, a more detailed description of the problem, plan of action, activities and project timeline are provided.

Following that are a few chapters which discuss the actual methodology used in the design and implementation of the tracking system. This is done with the aim that reproducing and extending the project can be done with minimal effort.

The results of testing the system are then shown.

Finally, the report ends off with conclusions on the project's success or failure, along with recommendations for future work.


\newpage

%-----------------------------------------------------------------------
%-----------------------------------------------------------------------
%-----------------------------------------------------------------------

\chapter{Literature Review}

This chapter provides an overview of the literature surrounding general tracking systems, as well as the components of the system designed in this report: object detection, camera gimbals, Extended Kalman Filters and controller design on a single board computer.

\section{Importance and applications of real-time computer-vision based autonomous tracking systems}
There are a number of applications for systems which track an object autonomously using a camera and computer-vision techniques alone. To unpack this sentence:

% TODO: consider changing to subsection?
\textit{Tracking systems:} a large number of systems which interact with the world must track an object. To give three examples, in our daily lives we as humans visually track other cars when driving, keep an idea of our geographic location and estimate the position and velocity of people as they walk so as to not bump into them. Examples more relevant to this project include pointing camera at a ball during a sports match and tracking poachers from the air in a game park.

\textit{Autonomous:} a human could (in principal) manually track an object using their own vision and natural object-recognition ability. However, this isn't feasible for many applications: this type of job can be fairly boring, resulting in the human losing concentration over time. Another issue is that the human must be paid to perform this task. This completely excludes all applications which must have a low running cost.

In contrast, autonomous tracking systems require little to no ongoing work, and thus can have vastly reduced running costs with higher reliability over long run times.

\textit{Real-time:} {\Large \color{red} not all systems run in realtime, but many applications require this}

\textit{Computer-vision based}: many tracking systems require a device to be placed on the object being tracked. Some still require a camera - for example, one could place an infra-red beacon on the object and pair it with a camera with an appropriate lens. Others involve no cameras, and instead use embedded sensors alone (such as GPS, accelerometers and gyrometers). Again, while this may be acceptable for some systems, it comes at a cost: some sort of device must be placed on the object. If the object is a cheetah, this may cause a disturbance. The device must be kept charged, so frequent interference with the object is required. Physical contact must be made with the object to install the tracker - this isn't always possible.

In contrast, object tracking based on computer vision doesn't require any device to be placed on the object. This is far more scalable, as the same system can track more objects without any additional hardware. In contrast, embedded tracker based systems require an additional piece of hardware for every object being tracked.

However, these benefits come at a cost: the object must be close enough to the camera system that the object can be recognised. In contrast, an embedded GPS, IMU and radio system could track objects kilometres away.

% https://en.wikipedia.org/wiki/Infra-red_search_and_track
{\Large \color{red} ALSO talk about vision-based, which don't use ML}
It is also worth discussing movement-based tracking system. As an example, 
can't differentiate between types of things very well. can't track all types of objects.

{\Large \color{red} ALSO discuss things like radar, which send off a ping}
good range, give away position, can't differentiate very well (only car vs human but not human vs dog or male vs female), not amazing resolution.

{\Large \color{red} maybe put this comparison in a table??}
ADD: not to say one is BETTER than the other, but rather that many many things need to be tracked and each approach should be considered.





\section{Related works and possible applications}
% possibilities: https://www.movidius.com/applications
There are a large number of applications for a system which can track objects in real time. Some include:

\begin{itemize}
\item A robot which can sense nearby humans and make sure to not hurt them, or decide on a course of action which depends on the type of object in front of it.
\item A small autonomous airplane which patrols the skies above a national park, and broadcasts the location of any potential poachers that it finds.
\item A camera system which automatically tracks a ball or centroid of the position all players, helping to record sports games.
\end{itemize}

Some drone companies have already implemented computer-vision based tracking systems into their products. One example is DJI, which incorporates autonomous human tracking as one of the features of their Mavic line of drones. Another example is eyecloud, a company that has developed a security system which covers a wider area than a single static camera could, and only sends an alert when it detects humans (but not dogs, which would be a false alarm).


\newpage

%-----------------------------------------------------------------------
%-----------------------------------------------------------------------
%-----------------------------------------------------------------------

\section{Research Design and Timeline}
\subsection{Research Problem}
narrow angle gopro with computer vision better than multiple wide angled gopros??
\subsection{Elaboration of the Plan of Action}
Do the minimum complete system first, then work on any extras, even if they're core to the idea of the project

Decided that training on cheetahs isn't core

\subsection{Description of activies, methods and software platforms}
\subsection{... validation}
\subsection{Project Timeline}


\newpage

%-----------------------------------------------------------------------
%-----------------------------------------------------------------------
%-----------------------------------------------------------------------

\section{System modelling and design}

\subsection{Object detection}
\subsubsection{Intro to neural networks}
why nn instead of other cv techniques? ans: can in princple make this neural net track ANY object with high accuracy, with no extra difficult design work (just open datasets or a couple hundred hand labelled images)

Explain concept of 'learning', image classification vs object detection

CNN, transfer learning, pace of advancements is quick

How modern data science works (difference between data scientist, who designs the neural network architecture, and machine learning engineer, who tends to take existing designs, possibly modify them, and then put them into production

\subsubsection{Movidius Neural Compute stick}
required to get classification speed from $\approx 1 Hz$ to $\approx 10 Hz$

Had great difficulties. Limited choice of nn framework (tensorflow object detection isn't supported) and neural network architectures

\subsubsection{Comparison of neural network architectures}
Why I chose MobileNet (designed for applications such as mine) (can afford the accuracy loss)\\
Why I chose SSD (instead of sliding window) $\rightarrow$ inference time

\subsection{Gimbal}
\subsubsection{Intro to gimbals + their purpose}

\subsubsection{Comparison of actuator types}
servos: easy to use, but shaky

\subsubsection{Gimbal design inspirations}
The primbal from thingieverse

Sylv's mount

Some changes to make it work for my application

\subsubsection{Gimbal controller}
Tuned PID controller

\subsubsection{Gimbal motors}
Can increase gopro spacing etc (motors are v v strong)


\subsection{Extended Kalman Filter}

\subsection{Controller design and implementation}


\section{System Integration and Testing}

\section{Results}
DONT DISCUSS THE RESULTS, JUST SHOW THEM!

\section{Discussion}
THIS IS THE ONLY SECTION WHERE THE RESULTS ARE DISCUSSED

\section{Conclusions}
test text

\section{Recommendations and Future Work}
Talk about: \\
- do other objects/animals \\
- predict distance with nn afterwards (using gopro footage)

\section{References}

\appendix
\section{Additional Files and etc etc}

\section{Addenda}
\subsection{Ethics Forms}

% ----------------------------------------------------------------
\end{document}
