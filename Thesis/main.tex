\documentclass{article}

\usepackage{graphicx}

%-----------------------------------------------------------------------
%-----------------------------------------------------------------------
%-----------------------------------------------------------------------

\begin{document}
\title{Alex Thesis Project}
\author{Alexander Knemeyer}

\maketitle

\pagenumbering{gobble}  % turn off page numbering

\newpage

\pagenumbering{roman}

\begin{abstract}
% [Explain the need/uses]
Many systems require a camera to identify the objects in it's field of view, and then move the camera to keep it aimed at a certain type of object. Examples of this need include filming cheetahs to further research into bio-mimicry, having a quadcopter record and follow a sportsperson as they move around and helping wildlife parks to spot and track poachers before they can cause harm.

% [Paragraph on computer vision]
Luckily, there has been a recent surge in performance for computer vision techniques. Neural networks often achieve higher accuracy scores than humans in tasks such as object detection. The technology is in general very accessible, with a strong culture of open-research and open-source.

% [Paragraph on raspberry pi + NCS]
However, image recognition using neural networks is an extremely compute intensive task. The nature of the computation suits parallelisation, but inferring a result from an underpowered mobile device (such as a raspberry pi, which is a type of single board computer) can take a full second or longer. Again, there exists a solution: neural accelerators, which are essentially a GPU on a USB stick. These enable object recognition in near real-time.

% [Paragraph on EKF]
Despite the speedups offered by neural accelerator sticks, the maximum number of inferences possible per second is still somewhat low. In this respect, the Kalman Filter can be useful. It enables the computer to predict where the tracked object will be in between updates from the neural network.

Thus, the aim of this project is as follows: using a modern neural network architecture, identify and track cheetahs from a mobile platform. The computing power is limited (a Raspberry Pi and Movidius Neural Compute Stick) which results in slow update rates. In addition, only position is measured. Thus, an Extended Kalman Filter is used to help ensure robust and smooth tracking of the target.
\end{abstract}

\newpage

%-----------------------------------------------------------------------
%-----------------------------------------------------------------------
%-----------------------------------------------------------------------

\section*{Terms of Reference}

%-----------------------------------------------------------------------
%-----------------------------------------------------------------------
%-----------------------------------------------------------------------

\section*{Acknowledgement}
thanks Callen and Amir \\
thanks Justin Pead \\
thanks Stacey for laser cutting \\
thanks family and friends

\newpage

%-----------------------------------------------------------------------
%-----------------------------------------------------------------------
%-----------------------------------------------------------------------

\tableofcontents
\newpage

\listoffigures
\listoftables

\newpage

\pagenumbering{arabic}

%-----------------------------------------------------------------------
%-----------------------------------------------------------------------
%-----------------------------------------------------------------------

\chapter{Introduction}
{\Large \color{red} have some text here? }
\section{Background to the study}

While the ability to autonomously track an object has a wide range of applications, it is useful to limit oneself to a particular need in order to produce quantifiable 'success or failure' metrics.

The UCT mechatronics laboratory's Rapid Acceleration and Manoeuvrability group studies cheetahs with the aim of reproducing their extraordinary manoeuvrability in two and four legged robots. For this, they require footage of cheetahs as they run. The cameras must be placed in pairs to achieve depth perception.

\vskip 15mm
{\Large \color{red} show top-down type image of multiple pairs of gopro cameras}
\vskip 15mm

Currently, they place multiple static pairs of cameras, each pair covering a different area, in order to continuously film an animal as it moves around. Having multiple cameras requires significant amounts of setup time, effort into synchronizing the footage, a total higher price as the cost of multiple cameras is expensive and extra work in keeping everything recharged. In addition, to maximise the time spent recording the animal, the cameras must be set to wide-angle. This introduces distortion into the recordings.


\section{Objectives of the study}
The objective of the study is thus to design and implement a system which can autonomously identify and track a fast-moving object using modern computer vision methods. A successful design would be cheaper overall, track the object for longer, and not require the cameras to be set to distorting wide-angle modes.


\section{Scope and limitations}
Due to the limited time and finances in an undergraduate thesis, the system will not be tested on actual cheetahs during the course of the project. Instead, it will be tested on humans and trained dogs with the hope that this sufficiently mimics the scenario of tracking cheetahs (or any other fast-moving object, for that matter).

However, the neural network will be retrained to be able to track cheetahs. This will be tested in simulation.

Finally, the study will be limited to tracking only one object in the frame at a time. This is because it is not immediately obvious what should be done when more than one object is present in the frame.


\section{Plan of development}
The development schedule was as follows:

\begin{enumerate}
\item Design the structure of the system as a whole.
\item Set up the main computing platform (a raspberry pi), installing relevant software and testing the camera module.
\item Choose a neural network architecture and deploy it onto the neural accelerator stick.
\item Design and 3D print a camera gimbal, and procure the necessary controller and actuators.
\item Rewrite the gimbal controller communication standard in python to allow the raspberry pi to send and receive commands.
\item Model the expected movement of the tracked object, and then design and implement a Kalman Filter (EKF).
\item Implement a parallel process-based data pipeline which takes a photo on the picam, preprocessess it, passes it through the neural accelerator, retrieves the results, passes the results into the EKF and then commands the gimbal motors to re-orient the camera appropriately.
\item Test final the system
\end{enumerate}

\section{Report outline}




\newpage

%-----------------------------------------------------------------------
%-----------------------------------------------------------------------
%-----------------------------------------------------------------------

\chapter{Literature Review}

\section{Importance and applications of ...}
this will be system-level stuff (general review of tracking objects) (will probably be centred around military and drone companies like DJI?) (also IR tracking, etc)

\section{Academic works related to ...}


\newpage

%-----------------------------------------------------------------------
%-----------------------------------------------------------------------
%-----------------------------------------------------------------------

\section{Research Design and Timeline}
\subsection{Research Problem}
narrow angle gopro with computer vision better than multiple wide angled gopros??
\subsection{Elaboration of the Plan of Action}
Do the minimum complete system first, then work on any extras, even if they're core to the idea of the project

Decided that training on cheetahs isn't core



\subsection{Description of activies, methods and software platforms}
python for pretty much all software \\
tensorflow, caffe and keras as DL frameworks \\
git to synchronize work done on pi, my laptop, VM and minibeast \\
extensive use of linux for development



\subsection{... validation}



\subsection{Project Timeline}
January/February: some intro to computer vision stuff. Preliminary research into NCS + initial setup on pi

June/July vac: (forgot exactly)

August/September: work on project

October: final tests, write up thesis, hand in. Also do poster and presentation



\newpage

%-----------------------------------------------------------------------
%-----------------------------------------------------------------------
%-----------------------------------------------------------------------

\section{System modelling and design}

\subsection{Object detection}
It was required that the system detect the position of an object from a photo. In particular, the object detector should,

\begin{itemize}
	\item not require a beacon or light to be placed on the object,
	\item run at around 5 to 10 Hz,
	\item make predictions with high accuracy,
	\item work regardless of the orientation of the object, and
	\item be able to track any object with minimal extra work/design required.
\end{itemize}

Convolutional Neural Networks (CNNs) can fill all these requirements. Other computer vision techniques (such as convolution with kernels to find certain techtures) generally run at higher speeds but are sorely lacking in terms of accuracy and robustness. Thinking and design work is required to make the object detector work with new types of objects, whereas CNNs simply require a few hundred or thousand labelled images and a few hours on a modern GPU. There are many open datasets which can be used to make this work easier.

\subsubsection{Intro to neural networks}
Machine learning, in which data is used to create a model which maps some input to an output, has recently seen a rapid increase in progress and popularity. Neural networks (a general structure for machine learning models) have played a large part in this sudden interest. The growth has been largely driven by Deep Neural Networks (DNNs) grown have come into the spotlight 

Explain concept of 'learning', image classification vs object detection

CNN, transfer learning, pace of advancements is quick

How modern data science works (difference between data scientist, who designs the neural network architecture, and machine learning engineer, who tends to take existing designs, possibly modify them, and then put them into production

\subsubsection{Movidius Neural Compute stick}
CNNs require a large number of operations to produce an output. The computation required depends on the neural network architecture, though even 'small' object detectors generally require at least two million Multiply-Accumulate instructions with significant amounts of data being loaded to and from the cache. Combined with the slower processor on the raspberry pi, the neural network would not be able to run in real time.

% nn and params: https://mxnet.apache.org/api/python/gluon/model_zoo.html

Luckily, this work can be parallelised. There are three types of parallelisation which tend to occur - during the application of $3 \times 3$ kernels, during matix multiplication and in parts of the neural network where the data flow is naturally parallel. This work is a natural fit for GPUs, though since GPU support for CNNs on the raspberry pi is lacking, other means had to be investigated.

% https://petewarden.com/2014/08/07/how-to-optimize-raspberry-pi-code-using-its-gpu/
% https://rpiplayground.wordpress.com/2014/05/03/hacking-the-gpu-for-fun-and-profit-pt-1/

This brought us to the Movidius Neural Compute Stick (NCS) - a specialized neural network accelerator which plugs into the raspberry pi (or any other computer). The NCS requires extra power and space, but its 12 purpose-built processing cores typically result in inference speed increases of between $700\%$ and $100\%$. It also has the benefit of being purpose built for CNNs.

% [ TODO: RESULTS: Had great difficulties. Limited choice of nn framework (tensorflow object detection isn't supported) and neural network architectures ]\\
% [ TODO: RESULTS: WENT FROM $\approx 1 Hz$ to $\approx 10 Hz$ ]

\subsubsection{Comparison of neural network architectures}
Not all neural network architectures are equal - generally, differences between models include:

\begin{itemize}
	\item classification accuracy
	\item inference speed, which is a function of the number of operations and number of weights
	\item data required for training/fine tuning
	\item the existence (or lack) of pretrained models in your specific framework
	\item whether recent innovations in the field have been included, and
	\item the underlying method in which objects are detected and localised within the image
\end{itemize}

Newer neural networks often take ideas from older architectures. Sometimes, they even include all or most of a previous architecture as part of the design of the new model. An example of this is MobileNet - an architecture designed at Google, aimed to run quickly on modern mobile devices (such as their newer smartphones). A common practice is to train MobileNet to classify objects on a given dataset, then remove the final layers, concatenate it with another model (with MobileNet acting as a feature extractor) and end up with an object detector.

Since MobileNet was designed for mobile devices, it traded some classification accuracy for performance. However, the architecture is remarkable in that the performance is significant while the classification accuracy is not. Thus, it was chosen as the feature detector for the project.

Next was the choice of the actual object detector. There are two (TODO: check this) main ideas for this approach: one could get an image classifier and apply it to the image multiple times (such as 25 times) resulting in a grid of overlapping detections.

[ SHOW IMAGE ]

Using the prediction probability for each part of the image, one can estimate where in the frame the desire object(s) is. This has the advantage of allowing for a simpler network architecture, but comes at the cost of performance (as the neural network must be applied multiple times) and resolution (determined by the grid size, which determines the run time). This is known as the 'sliding window' technique.

The other approach uses the fact that the neural network can locate objects in the frame, and thus returns the locations of objects in the image embedded in the output nodes. An example of this type of network is the Single Shot Detector (SSD) which, as the name suggests, requires only a single inference to output a list of objects, their classification probabilities and their locations in the frame.

[ TODO: RESULTS: talk about trying to make my own network with two output nodes, which simply predicts the centroid of the object as a scale from -1 to 1, with -1 being the bottom or left side of the image and 0 being the center. Couldn't get it to compile due to tensorflow support ]

\subsubsection{Comparison of neural network frameworks}
% speeds: https://arxiv.org/pdf/1608.07249v7.pdf
Tensorflow - low level, annoying and hard to use, fast
Keras - simple to use, tensorflow as backend so relatively fast, less fine grained control possible, can't find all the latest models there, can export model as tensorflow
Caffe - fast, easy to use, hard to extend, only really for CNNs, might be dying out

% ------------------------------------------------------------------------------------------------
% ------------------------------------------------------------------------------------------------
% ------------------------------------------------------------------------------------------------

\subsection{Gimbal}
Once an object has been found and a desired control action calculated, the camera must be moved. This required the use of a camera gimbal.

\subsubsection{Comparison of actuator types}
The type of actuator used in the gimbal has a significant effect on the design of the gimbal, the speed it can move, the camera weight it can handle and quality of the resulting photos and videos. Three main actuator types were considered and compared:

DC motors: quite long and heavy, require encoders or IMU, potentiallty messy circuitry (for bidirectional), not extremely fast/efficient

Servo motors: very easy to use (essentially a DC motor with circuitry built in), somewhat large, good torque, can work without encoders/IMU though not necessarily advised, get very jittery control

Brushless DC (BLDC) motors: quite complex to use, somewhat expensive, very compact, very fast and give very smooth footage. Also, the gimbal controller (which is unfortunately a requirement) makes things easier and very professional. Requires either encoders or IMU

Chose BLDC as the motors are small, and it has the highest capacity for "good" footage (silky smooth operation with fast responses)

\subsubsection{Gimbal design inspirations}
I began by looking for gimbal designs on free online repositories. Couldn't find one that suited my needs, so to avoid re-inventing the wheel I used the primbal from thingieverse as a base and took some inspiration from Sylv's mount (such as how to mount to the specific quad).

Had to make some changes to make it work for my application

\subsubsection{Gimbal controller}
Tuned PID controller

Rewrote SerialAPI in Python (took a while) (releasing for free online)

\subsubsection{Gimbal motors}
Can increase gopro spacing etc (motors are v v strong)

% ------------------------------------------------------------------------------------------------
% ------------------------------------------------------------------------------------------------
% ------------------------------------------------------------------------------------------------

\subsection{Extended Kalman Filter}

% ------------------------------------------------------------------------------------------------
% ------------------------------------------------------------------------------------------------
% ------------------------------------------------------------------------------------------------

\subsection{Controller design and implementation}


\section{System Integration and Testing}

\section{Results}
DONT DISCUSS THE RESULTS, JUST SHOW THEM!

\section{Discussion}
THIS IS THE ONLY SECTION WHERE THE RESULTS ARE DISCUSSED

\section{Conclusions}
test text

\section{Recommendations and Future Work}
Talk about: \\
- do other objects/animals \\
- predict distance with nn afterwards (using gopro footage)

\section{References}

\appendix
\section{Additional Files and etc etc}

\section{Addenda}
\subsection{Ethics Forms}

% ----------------------------------------------------------------
\end{document}
