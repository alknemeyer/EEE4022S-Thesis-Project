\chapter{Conclusions}

Based on the test results shown in Chapter~\ref{chap:results} and discussion in Chapter~\ref{chap:discussion}, it can be conclusively said that the tracking camera system works better than a stationary camera setup. This is due to three main reasons:

First, the maximum field of view of the tracking system is significantly greater than that of a single stationary camera. One could use multiple stationary cameras, but this comes at a higher cost and extra work on the part of the researchers.

Second, the distortion which arises from setting the stationary cameras to wide-angle mode is significantly higher than the distortion from setting the tracking cameras to an narrow angle mode, which is made feasible by the tracking system. One could counter this buy purchasing even larger numbers of stationary cameras and setting each to the less distorting narrow field of view, but this comes at a higher cost and with extra work on the part of the researchers.

Third, there are other benefits to having a system which aims at a target which aren't practically feasible using a stationary camera setup. For instance, one could attach one or more distance sensors and a GPS to the gimbal to estimate the absolute location of the object.

When viewed as a general tracker in its own right, the system is not perfect and could see significant benefits given an extra week or two of time. The study was also not conclusive in that the tests in the laboratory alone aren't enough to empirically prove that the system could track a fast moving cheetah (given a neural network which can detect one) running close to the camera with a high perceived angular velocity.
