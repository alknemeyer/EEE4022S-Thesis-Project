\chapter{The Gimbal}
Given an angle to a detected object, the camera system and distance sensor must be able to rotate so as to track the object. This requires the use of an actuated gimbal capable of moving this payload.


\section{The gimbal frame}
There are already a plethora of gimbal systems available for purchase, and gimbal designs available for free use. However, the requirements of the gimbal narrow down the options significantly. Gimbals typically need only keep a single small camera level, or rotate according to commands from an RC system. This project requires a gimbal which has space for two cameras spaced a few centimetres apart and can rotate according to commands from a Raspberry Pi. This necessitated the design of a new system.

While they couldn't be used without modification, the designs mentioned subsection \ref{ssec:gimbal_design_inspirations} were a good enough reference point that a new system could be created with relative ease.

Some priorities for the new gimbal design were as follows:
%
\begin{itemize}
\item It should be possible make the frame using equipment available at the university, such as a laser cutter or 3D printer.
\item It should be straightforward for someone to modify the design to house a new gadget.
\item In order to minimize the strain on actuators and the control system, there should a mechanism to help balance the gimbal.
\item As a result of the cameras being placed a few centimetres, there is an unbalanced torque. The effects of this should be minimized.
\end{itemize}
%
Based on these criterion, a gimbal was designed using SolidWorks, a CAD software used frequently in engineering design. A render of the final model used for this project is shown below, in Figure~\ref{fig:assembled_render}.

\begin{figure}[h!]
  \centering
  \includegraphics[width=\textwidth]{methodology/assembled_render}
  \caption{\label{fig:assembled_render} A render of the gimbal designed for this project.}
\end{figure}

All of the priorities mentioned earlier were met: the top mounting panels are easily made using a laser scanner, while the blue components are straightforward to make with a 3D printer. The payload can be modified without needing to extend and reprint the mounting brackets. This also has the effect of making it easy to move the mass around to balance the gimbal. Finally, the addition of a support bracket remove the effect of a potentially large torque from two cameras.

{\Huge \color{red} show \\ \\ \\ \\ \\ pictures \\ \\ \\ \\ \\ of \\ \\ \\ \\ \\ this \\ \\ \\ \\ \\ stuff \\ \\ \\ \\ \\ and \\ \\ \\ \\ \\ discuss \\ \\ \\ \\ \\ aspects \\ \\ \\ \\ \\ of \\ \\ \\ \\ \\ the \\ \\ \\ \\ \\ design!}
{\color{red} maybe also do text next to the pics}
%https://tex.stackexchange.com/questions/119799/text-next-to-image



\section{The gimbal controller}
Properly controlling BLDC motors to stabilize a gimbal is an entire project on its own. Luckily, there are well-designed existing products which easily fulfil the requirements needed. While research was made into the strengths and weakness of the various options available, factors such as local availability, price and time until reception of the gimbal resulted in a single actual choice: purchasing a second hand BaseCam Electronics Simple Brushless Gimbal Controller (SBGC) 32-bit controller from someone who used it for drone footage.

The controller is powered off a 8-25V source and has built in power electronics, simplifying the number of components required. Up to three BLDC gimbal motors can be powered off the device.

The angular state of the gimbal is estimated using an IMU placed near the camera. The IMU contains an accelerometer and a gyroscope.

%\subsection{The Serial API}

It also has an entire serial API which allows an external computer to write commands to and read information from the gimbal. Example commands include setting motor angles and rotation rates, while those same variables can be read from the controllers built in Kalman Filter (which estimates those states).

The controllers serial API is specified in a long {\color{red} xxx} page document. The manufacturer also provides an Arduino implementation of the serial API. However, since the rest of the project was implemented in python, it was decided that the relevant communication protocol would be re-implemented in python. In addition, since a variety of users on the BaseCam forums have mentioned that they would appreciate having access to such a python module, the library will be released for free on the forums.

The serial API was made to work by looking through the serial API specification, reading the code in the Arduino implementation and finally scoping the output of the Raspberry Pi as a last resort debugging technique.


\section{Integrating components}
Once the frame had been 3D printed, and the controller and motors purchased, the gimbal was ready for assembly and tuning.

\subsection{Gimbal motors}
Four LD-Power 2208 gimbal motors were included in the purchase of the gimbal controller. Due to the small market for gimbals, most lower-speed, higher-torque 'gimbal motors' are simply rewound high-speed quadcopter/airplane motors.

The amount of torque required from a gimbal motor tends to depend more on the balance of the frame and friction in the bearings than other factors, like the rotational inertia of the objects being rotated.

\subsubitem{Balancing the system}
talk about moving things along until the cameras would hold any angle without turning

\subsection{Tuning the controller}
selected auto-PID but this didn't work

thus, tuned them in order P $\rightarrow$ I $\rightarrow$ D $\rightarrow$ repeat.

\subsection{Testing the serial API}



