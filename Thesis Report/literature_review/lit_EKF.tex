\section{The Kalman Filter}
It is often desired to get a better estimate of a system's state than a sensor can achieve on its own. In these scenarios, it is useful to consider the use of a Kalman Filter or its nonlinear variant, the Extended Kalman Filter.

\subsection{An brief introduction to the Kalman Filter}
A Kalman Filter \cite{kalman1960new} is an algorithm used to estimate the state of a linear system ($\hat{x}$) along with the uncertainty of the estimate ($P$) based on measurements over time ($z$). The measurements are assumed to include noise with covariance $R$. It makes use of a model of the system ($F$) which represents the dynamics of the system ($\hat{x}_{k} = F\hat{x}_{k-1}$), along with a model which represents inaccuracies in the model, dubbed 'process noise' ($Q$) and measured factors which affect the state, but aren't part of the current set of states ($u$). It takes into account the sensor scaling which maps from between sensor measurements and state values ($H$).

A variety of excellent resources exist which explain the a deeper intuition behind the Kalman Filter \cite{website:wlu_kalman_tutorial, website:bzarg_kalman_tutorial}. Thus, only the equations and high-level concepts which are absolutely necessary will be discussed in this chapter.

Kalman Filters continuously alternate between a predict stage and an update stage. In the predict stage, the Kalman Filter uses a model of the system and known disturbances to update the state estimate and uncertainty estimate. These are represented by the equations,

\[ \hat{x}_k = F_k \hat{x}_{k-1} + B_k u_k \]
\[ P_k = F_k P_{k-1} F_k^T + Q_k \]

The $k$ subscripts denote the point in time at which the calculation takes place. This means that any matrix in the algorithm may be updated as time goes on.

In the 'update' stage, the Kalman Filter makes use of a model of uncertainty estimates to produce a new state estimate which is a trade-off between new data from a sensor, and the state estimate produced by the predict stage.

\[ \hat{x}_k = \hat{x}_{k-1} + K_k (z_k - H_k x_{k-1}) \]
\[ P_k = (I - K_k H_k) P_{k-1} \]

where the Kalman gain $K$ at a time $k$ is calculated as,
\[ K_k = P_k H_k^T (H_k P_k H_k^T + R_k)^{-1} \]

%These stages are alternated as \emph{predict} $\rightarrow$ \emph{update} $\rightarrow$ \emph{predict} $\rightarrow$ \dots
As an illustration, consider the extremes during the update stage: when $K = 0$, we have

\[ \hat{x}_k = \hat{x}_{k-1} + 0 (z_k - H_k x_k) = \hat{x}_{k-1} \]

If $H_k = 1$ and $K = 1$, we have

\[ \hat{x}_k = \hat{x}_{k-1} + (z_k - x_k) = z_k \]

Values of $K$ in between result in a state update which is some combination of the sensed values and modelled dynamics of the system. Thus, the role of the Kalman Filter is essentially to find the optimal trade off between trusting the model and trusting the sensor readings.

Finally, given a Kalman Filter with an accurate estimation of systems state and a model which adequately predicts the future state of the system, one can interpolate between readings from a sensor with slow update rates. This could mean getting model updates at a required speed of 20 Hz using a sensor which only makes measurements at a rate of 5 Hz.



\subsection{An introduction to the Extended Kalman Filter}
An Extended Kalman Filter (EKF) is simply a nonlinear version of a Kalman Filter, where the transition from estimated state $\hat{x}_{i-1}$ to state $\hat{x}_i$ is some \emph{nonlinear} mapping $f(x, u)$ and the mapping from state values to sensor values is a nonlinear function $h(x)$. It works by using a Taylor approximation to linearize about an estimate of the current mean and covariance.

Therefore, implementation-wise, the only difference between the linear and non-linear variants of the Kalman Filter is that the state transition and observation models can be non-linear differentiable functions:

\[ F \cdot x \rightarrow f(x) \qquad H\cdot x \rightarrow h(x) \]

This also results in differing values for their respective Jacobian matrices. This has importance knock-on effects for how much each part of the model or sensor should be trusted. As an example, a Kalman Filter may trust a sensor reading which indicates a high value less than a reading which indicates something lower due to the gradient of the sensing function.
