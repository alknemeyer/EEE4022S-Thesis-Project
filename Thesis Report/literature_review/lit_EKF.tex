\section{Extended Kalman Filter}
\subsection{An introduction to the Kalman Filter}
A Kalman Filter \cite{kalman1960new} is an algorithm used to estimate the state of a linear system ($\hat{x}$) along with the uncertainty of the estimate ($P$) based on noisy measurements over time ($z$, with noise covariance $R$). It makes use of a model of the system ($F$) which represents the dynamics of the system ($\hat{x}_{k} = F\hat{x}_{k-1}$, with process noise $Q$) and measured factors which affect the state, which aren't part of the current set of states ($u$). It takes into account the sensor measurement which maps from sensor measurements to state values ($H$).

A variety of excellent resources exist which explain the a deeper intuition behind the Kalman Filter \cite{website:wlu_kalman_tutorial, website:bzarg_kalman_tutorial}. Thus, only the relevant equations will be discussed in this chapter.

In the 'predict' stage, the Kalman Filter uses a model of the system and known disturbances to update the state estimate and uncertainty estimate. These are represented by the equations,

\[ \hat{x}_k = F \hat{x}_{k-1} + B u \]
\[ P_k = F_k P_{k-1} F_k^T + Q_k \]

In the 'update' stage, the Kalman Filter makes use of a model of uncertainty estimates to produce a new state estimate which is a trade-off between new values from a sensor, and the state estimate produced by the predict stage.

\[ \hat{x}_k = \hat{x}_{k-1} + K_k (z_k - H_k x_k) \]
\[ P_k = (I - K_k H_k) P_{k-1} \]
where the Kalman gain $K$ at a time $k$ is calculated as,
\[ K_k = P_k H_k^T (H_k P_k H_k^T + R_k)^{-1} \]

These stages are alternated as \emph{predict} $\rightarrow$ \emph{update} $\rightarrow$ \emph{predict} $\rightarrow$ \dots As an illustration, after running the predict stage, consider the extremes: when $K = 0$, we have $\hat{x}_k = \hat{x}_{k-1} + 0 (z_k - H_k x_k) = \hat{x}_{k-1}$. If $H_k = 1$, whe $K = 1$, we have $\hat{x}_k = \hat{x}_{k-1} + (z_k - x_k) = z_k$. Values in between result in a state update which is some combination of the sensed values and modelled dynamics of the system. The Kalman Filter aims to find the optimal combination.

\subsection{An introduction to the Extended Kalman Filter}
An Extended Kalman Filter (EKF) is simply a nonlinear version of a Kalman Filter, where the transition from estimated state $\hat{x}_{i-1}$ to state $\hat{x}_i$ is some \emph{nonlinear} mapping $f(x, u)$ and the mapping from state values to sensor values is a nonlinear function $h(x)$. It works by using a Taylor approximation to linearize about an estimate of the current mean and covariance.

Implementation-wise, the only difference between the linear and non-linear variants is that the state transition and observation models can be non-linear differentiable functions:

\[ F \cdot x \rightarrow f(x) \qquad H\cdot x \rightarrow h(x) \]

This also results in differing values for their respective Jacobian matrices.
