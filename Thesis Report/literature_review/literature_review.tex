\chapter{Literature Review}

This chapter provides an overview of the literature surrounding general tracking systems, as well as the components of the system designed in this report: object detection, camera gimbals, Extended Kalman Filters and controller design on a single board computer.

\section{Importance and applications of real-time computer-vision based autonomous tracking systems}
There are a number of applications for systems which track an object autonomously using a camera and computer-vision techniques alone. To unpack this sentence:

% TODO: consider changing to subsection?
\textit{Tracking systems:} a large number of systems which interact with the world must track an object. To give three examples, in our daily lives we as humans visually track other cars when driving, keep an idea of our geographic location and estimate the position and velocity of people as they walk so as to not bump into them. Examples more relevant to this project include pointing camera at a ball during a sports match and tracking poachers from the air in a game park.

\textit{Autonomous:} a human could (in principal) manually track an object using their own vision and natural object-recognition ability. However, this isn't feasible for many applications: this type of job can be fairly boring, resulting in the human losing concentration over time. Another issue is that the human must be paid to perform this task. This completely excludes all applications which must have a low running cost.

In contrast, autonomous tracking systems require little to no ongoing work, and thus can have vastly reduced running costs with higher reliability over long run times.

\textit{Real-time:} {\Large \color{red} not all systems run in realtime, but many applications require this}

\textit{Computer-vision based}: many tracking systems require a device to be placed on the object being tracked. Some still require a camera - for example, one could place an infra-red beacon on the object and pair it with a camera with an appropriate lens. Others involve no cameras, and instead use embedded sensors alone (such as GPS, accelerometers and gyrometers). Again, while this may be acceptable for some systems, it comes at a cost: some sort of device must be placed on the object. If the object is a cheetah, this may cause a disturbance. The device must be kept charged, so frequent interference with the object is required. Physical contact must be made with the object to install the tracker - this isn't always possible.

In contrast, object tracking based on computer vision doesn't require any device to be placed on the object. This is far more scalable, as the same system can track more objects without any additional hardware. In contrast, embedded tracker based systems require an additional piece of hardware for every object being tracked.

However, these benefits come at a cost: the object must be close enough to the camera system that the object can be recognised. In contrast, an embedded GPS, IMU and radio system could track objects kilometres away.

% https://en.wikipedia.org/wiki/Infra-red_search_and_track
{\Large \color{red} ALSO talk about vision-based, which don't use ML}
It is also worth discussing movement-based tracking system. As an example, 
can't differentiate between types of things very well. can't track all types of objects.

{\Large \color{red} ALSO discuss things like radar, which send off a ping}
good range, give away position, can't differentiate very well (only car vs human but not human vs dog or male vs female), not amazing resolution.

{\Large \color{red} maybe put this comparison in a table??}
ADD: not to say one is BETTER than the other, but rather that many many things need to be tracked and each approach should be considered.





\section{Related works and possible applications}
% possibilities: https://www.movidius.com/applications
There are a large number of applications for a system which can track objects in real time. Some include:

\begin{itemize}
\item A robot which can sense nearby humans and make sure to not hurt them, or decide on a course of action which depends on the type of object in front of it.
\item A small autonomous airplane which patrols the skies above a national park, and broadcasts the location of any potential poachers that it finds.
\item A camera system which automatically tracks a ball or centroid of the position all players, helping to record sports games.
\end{itemize}

Some drone companies have already implemented computer-vision based tracking systems into their products. One example is DJI, which incorporates autonomous human tracking as one of the features of their Mavic line of drones. Another example is eyecloud, a company that has developed a security system which covers a wider area than a single static camera could, and only sends an alert when it detects humans (but not dogs, which would be a false alarm).
