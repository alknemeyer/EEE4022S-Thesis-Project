\chapter{Discussion}\label{chap:discussion}

Some things should be noted when considering the results shown previously.

The first is that, unless one decreased the effects of the image recognition latency, an angle between the cameras orientation and the object will always remain. This can be diminished if one has better knowledge of the animals movement prior to the experiment - for example, if it was known that the only thing being tracked is an object moving from left to right, the yaw Kalman Filter would be able to more reliably predict the current position of the object based on delayed data. This could result in better tracking, but comes with its own problems (such as overshoot if the object slows down, which is part of the RAM groups tests).

The second issue to discuss is that of the maximum angular range of the system, which is limited only by the wiring connecting the stationary part of the tracker to the rotating parts. While the tracker itself has a maximum angular rotation width of 175\textdegree, a portion of the field of view of the camera could be added to this to represent the total range that can be filmed. It may not make sense to add the full width, since part of the premise of the tracker is that object is aimed at directly.

Another topic is the tracking speed of the gimbal, and how it failed to track objects with high a velocity or acceleration. The Kalman Filters noise parameter can be adjusted to vary the state estimate from smoothness as a priority (which could be interpreted as seeing sudden changes in the sensed position as noise, which should be rejected) to robust tracking as a priority (where a sudden change in state is seen as the object moving quickly). In addition, the system would sometimes be unable to track a fast moving object simply due to the gimbals slower rotational rate, which wasn't investigated in time. It is unclear how much the Kalman Filter's parameters and gimbals slow movement limited the performance in this regard.

Finally, all of the tests were performed in a relatively confined space tracking close objects. It is unclear how well the tracker would perform tracking faster objects further away. Based on the results of this report, one could estimate its performance using the formula $\omega = v/r$, where $\omega$ is the rotational velocity, $v$ is the tangential angular velocity and $r$ is the distance to the object. Since the object tracker is confined to track objects at a certain maximum angular rate, but objects themselves move with linear velocities, as long as $\omega$ remains constant between tests on a human in a lab and a cheetah outdoors it is likely that the system would continue to track well.
