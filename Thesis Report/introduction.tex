\chapter{Introduction}
This report details the research, design, implementation and testing of an automatic object detection and tracking system. It was made as a part of a final year undergraduate project.

\section{Background to the study}
For years, some researchers have turned to bio-mimicry for insight and inspiration into the design of robots. One such example is the University of Cape Town (UCT) mechatronics laboratory's Rapid Acceleration and Manoeuvrability group, which studies cheetahs with the aim of reproducing their extraordinary manoeuvrability in two and four legged robots \cite{website:uct_mechatronics}. To understand the dynamics of cheetah movement, they require footage of animals as they start and stop a sprint. A requirement of their research is that the cameras must be placed in pairs to achieve depth perception~\cite{website:depth_perception_from_stereo_camera}.

The researchers currently place multiple static cameras, with each pair covering a different area, in order to continuously film an animal through a full sprint. An illustration of this can be seen in Figure~\ref{fig:multiple_gopro_pairs}. \\

\begin{figure}[h!]
  \centering
  \includegraphics[width=\textwidth]{multiple_gopro_pairs}
  \caption{\label{fig:multiple_gopro_pairs} An illustration of the camera set up in a typical animal movement study. Cheetah clipart from Change.org \cite{website:pic_of_cheetah}.}
\end{figure}

Unfortunately, this setup is not without its disadvantages: having multiple cameras requires extra setup time, effort into synchronizing the footage, money spent on equipment and extra work in keeping all devices charged. To maximise the time spent recording a sprint, the cameras are usually set to wide-angle mode. This introduces distortion into the recordings \cite{website:camera_distorion}.

This resulted in the following question: what if the cameras could instead rotate to track the cheetah as it moves? This way only a single pair of cameras would be required. An illustration of this in two moments of time is shown in Figure~\ref{fig:single_gopro_pair}.

\begin{figure}[h!]
  \centering
  \includegraphics[width=\textwidth]{single_gopro_pair}
  \caption{\label{fig:single_gopro_pair} The proposed solution: a single pair of cameras tracking an animal as it moves.}
\end{figure}

The researchers specified that such a tracking system should,

\begin{itemize}
	\item Not require a beacon, light or any other object to be placed on the object,
	\item Locate the object with high accuracy,
	\item Be able to differentiate between classes of objects, while treating different instances of the same class equally (in other words, locate \emph{any} cheetah regardless of its specific spot pattern),
	\item Have a low enough latency that the object can't escape the field of view of the cameras before a result is reached,
	\item Use a provided Raspberry Pi model 3 B as the main computing platform so as to remain mobile,
	\item Work robustly - this means locating the regardless of its orientation, lighting, specific shape, background, and so on - and finally,
	\item Be able to track any other object with minimal extra work or design required, should the focus of research be changed.
\end{itemize}

After comparing existing object tracking technologies, it was decided that the only way to robustly track the cheetahs in this scenario would be to use a neural network. This decision is elaborated on in section \ref{sec:compare_cv_techniques} of this report. The comparison showed that only machine learning techniques can accurately discriminate between a range of classes of objects using affordable and accessible sensing hardware alone (such as low-cost cameras).


\section{Other object tracking systems}
There are a large number of applications for a system which can track objects in real time using computer vision techniques. Some include:

\begin{itemize}
\item A robot which can sense nearby humans and make sure to not hurt them, or decide on a course of action which depends on the type of object in front of it.
\item A small autonomous airplane which patrols the skies above a national park, and broadcasts the location of any potential poachers that it finds.
\item A camera system which automatically tracks a ball or the centroid of the position all players, helping to record sports games.
\end{itemize}

Object tracking systems have already been incorporated into existing commercial products. One example is the "Mavic" drone from DJI, which can autonomously track humans \cite{website:DJI_mavic}. This allows people to launch the drone and have it autonomously follow them as they move around.

Another example is EyeCloud, a company that has developed a security system which covers a wider area than a single static camera could, and only sends an alert when it detects humans \cite{website:eyecloud}. This prevents false alarms from sounding when safe classes of objects, such as dogs, enter the field of view.



\section{Objectives of the study}
Based on the problem stated by the UCT researchers, the objective of the study is to design, build and test a system which can autonomously identify and track a fast-moving object using modern computer vision methods. A successful design would be able to rotate a pair of cameras to continuously aim at an object, track the object for longer than a stationary camera pair could, and not require the cameras to be set to distorting wide-angle modes.

Note that, while the ability to autonomously track an object has a wide range of applications, it is useful to limit oneself to a particular need in order to produce quantifiable 'success or failure' metrics. Thus, this project and report are mainly about solving a specific issue, with the premise that the project can be used for other applications with minimal modification. Discussions on literature and other applications will treat the system as a general purpose object tracker.



\section{Scope and limitations}
Due to the limited time and finances in an undergraduate thesis, testing the tracking system on actual cheetahs was not part of the agreed scope of this project. Instead, it was tested on moving humans in complex environments with the hope that this sufficiently mimics the scenario of tracking cheetahs (or any other fast-moving object) in the wild.

Additionally, the study was limited to tracking only one object in the frame at a time. This is because it is not immediately obvious what should be done when more than one object is present in the frame.


\section{Plan of development}
A development schedule was devised to decide how much time could be allotted for each component of the final system. It was as follows,

\begin{enumerate}
\item Create a rough design for the system as a whole to understand which skills and components would be necessary.
\item Spend a month learning relevant new theory, mostly in the field of computer vision.
\item Set up the main computing platforms, installing relevant software and testing the camera module, over the course of a week.
\item Optimize a cutting edge neural network for the specific scenario in this project over the course of two weeks. Failing that, choose an existing neural network architecture.
\item Compile the neural network and deploy it onto a neural accelerator stick, writing appropriate software, over the course of a week.
\item Use two weeks to design and 3D print a camera gimbal, and procure the necessary controller and actuators.
\item At the same time, rewrite the gimbal controller communication standard in python to allow the raspberry pi to send and receive commands.
\item Spend a week modelling the expected movement of the tracked object, and designing and implementing an appropriate Kalman Filter. Update this based on preliminary test results.
\item Spend two weeks implementing a parallel process-based data pipeline which takes a photo on the Pi camera, pre-processess it, passes it through the neural accelerator, retrieves the results, passes the results into the Kalman Filter and then commands the gimbal motors to re-orient the camera appropriately. 
\item Test final the system over the course of a week, tweaking the design based on results.
\item Use the last three weeks to write a report which summarizes the entire project, based on small notes taken during the course of the project.
\end{enumerate}



\section{Report outline}

The report begins with a review of the current literature surrounding object tracking. As the system combines a number of technologies, it also presents previous knowledge on all of the components which ultimately formed a part of final system.

Next, a more detailed description of the problem, plan of action, activities and project timeline are provided.

Following that are a few chapters which discuss the actual methodology used in the design and implementation of the tracking system. This is done with the aim that reproducing and extending the project can be done with minimal effort.

The results of testing the system are then shown.

Finally, the report ends off with conclusions on the project's success or failure, along with recommendations for future work.
