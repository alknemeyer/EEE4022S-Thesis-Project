\chapter{Introduction}
{\Large \color{red} have some text here? }
\section{Background to the study}

While the ability to autonomously track an object has a wide range of applications, it is useful to limit oneself to a particular need in order to produce quantifiable 'success or failure' metrics.

The UCT mechatronics laboratory's Rapid Acceleration and Manoeuvrability group studies cheetahs with the aim of reproducing their extraordinary manoeuvrability in two and four legged robots. For this, they require footage of cheetahs as they run. The cameras must be placed in pairs to achieve depth perception.

\vskip 15mm
{\Large \color{red} show top-down type image of multiple pairs of gopro cameras}
\vskip 15mm

Currently, they place multiple static pairs of cameras, each pair covering a different area, in order to continuously film an animal as it moves around. Having multiple cameras requires significant amounts of setup time, effort into synchronizing the footage, a total higher price as the cost of multiple cameras is expensive and extra work in keeping everything recharged. In addition, to maximise the time spent recording the animal, the cameras must be set to wide-angle. This introduces distortion into the recordings.


\section{Objectives of the study}
The objective of the study is thus to design and implement a system which can autonomously identify and track a fast-moving object using modern computer vision methods. A successful design would be cheaper overall, track the object for longer, and not require the cameras to be set to distorting wide-angle modes.


\section{Scope and limitations}
Due to the limited time and finances in an undergraduate thesis, the system will not be tested on actual cheetahs during the course of the project. Instead, it will be tested on humans and trained dogs with the hope that this sufficiently mimics the scenario of tracking cheetahs (or any other fast-moving object, for that matter).

However, the neural network will be retrained to be able to track cheetahs. This will be tested in simulation.

Finally, the study will be limited to tracking only one object in the frame at a time. This is because it is not immediately obvious what should be done when more than one object is present in the frame.


\section{Plan of development}
The development schedule was as follows:

\begin{enumerate}
\item Design the structure of the system as a whole.
\item Set up the main computing platform (a raspberry pi), installing relevant software and testing the camera module.
\item Choose a neural network architecture and deploy it onto the neural accelerator stick.
\item Design and 3D print a camera gimbal, and procure the necessary controller and actuators.
\item Rewrite the gimbal controller communication standard in python to allow the raspberry pi to send and receive commands.
\item Model the expected movement of the tracked object, and then design and implement a Kalman Filter (EKF).
\item Implement a parallel process-based data pipeline which takes a photo on the picam, preprocessess it, passes it through the neural accelerator, retrieves the results, passes the results into the EKF and then commands the gimbal motors to re-orient the camera appropriately.
\item Test final the system
\end{enumerate}

\section{Report outline}


